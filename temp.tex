\documentclass[12pt]{article}
\begin{document}

 WICHTIGE BEGRIFFE ZU KATEGORIEN\\

- das ist eine testzeile \\
- C, D Kategorien, $F : C \mapsto D$ Funktor, $B \subset C$, $a,b,s,q,x,z \in Obj(C)$, $f : a \mapsto b$ \\
\newline
- B heisst vollstaendige Unterkategorie von C, falls
\begin{description}
    \item[i) $Obj(B) \subset Obj(C)$]
    \item[ii) $\forall X,Y \in Obj(B): Mor_B(X,Y) = Mor_C(X,Y)$]  
\end{description}
- Morphismus $m : a \mapsto b$ monomorph, falls
\begin{description}
    \item[i) $fuer \: g_1,g_2 : x \mapsto a \: folgt \: aus \: m \circ g_1 = m \circ g_2 \Rightarrow g_1 = g_2$] 
\end{description}
- Morphismus $e : a \mapsto b$ epimorph, falls
\begin{description}
    \item[i) $fuer \: g_1,g_2 : b \mapsto x \: folgt \: aus \: g_1 \circ e = g_2 \circ e \Rightarrow g_1 = g_2$] 
\end{description}
- Morphismus $k : d \mapsto a$ Kern, falls
\begin{description}
    \item[i) $f \circ k = 0$]
    \item[ii) $zu \: h : x \mapsto a \: \exists ! h' : x \mapsto d: h = k \circ h'$] 
\end{description}
- Morphismus $k : b \mapsto d$ Kokern, falls
\begin{description}
    \item[i) $k \circ f = 0$]
    \item[ii) $zu \: h : b \mapsto x \: \exists ! h' : d \mapsto x: h = h' \circ k$] 
\end{description}
- Funktor F heisst volltreu, falls 
\begin{description}
    \item[i) $Mor_C(a,b) \mapsto Mor_D(F(a),F(b)), f \mapsto F(f) \: bijektiv$] 
\end{description}
- Funktor F heisst Aequivalenz von kategorien, falls
\begin{description}
    \item[i) $\exists G : D \mapsto C, \: sodass \: F \circ G \cong id_D \: und \: G \circ F \cong id_C$] 
\end{description}

 WAS IST EINE ABELSCHE KATEGORIE\\

{\large Sei A Kategorie. A heisst abelsch, falls $\forall X,Y,Z \in$ Obj(A):}

\begin{description}
    \item[i) $\forall f,f_1,f_2 \in Hom_A(X,Y), g,g_1,g_2 \in Hom_A(Y,Z):$]
        \[g \circ (f_1 + f_2) = g \circ f_1 + g \circ f_2\]
        \[(g_1 + g_2) \circ f = g_1 \circ f + g_2 \circ f\]
    \item[ii) $\forall X_1,X_2 \exists X_1 \otimes X_2 \: mit \: Morphismen \: p_v : X_1 \otimes X_2 \mapsto X_v \: und \: i_v : X_i \mapsto X_1 \otimes X_2:$]
        \[p_v \circ i_v = id_{X_v}\]
        \[i_1 \circ p_1 + i_2 \circ p_2 = id_{X_1 \otimes X_2}\]
    \item[iii) $\exists \: Nullobjekte$]
    \item[iv) $\exists \: Kerne,\: Kokerne$]
    \item[v) $Jeder \: Monomorphismus \: ist \: Kern, \: jeder \: Epimorphismus \: ist \: Kokern$]   
\end{description}
    
 WAS SIND KETTENKOMPLEXE\\

- das ist eine testzeile \\
- Familie $C_n$ von Objekten aus abelschen Kategorie mit Morphismen $d_n : C_n \mapsto C_{n-1}$, sodass $d_n \circ d_{n+1} : C_{n+1} \mapsto C_{n-1} = 0$ \\
- $H_n(C) := ker(d_n) / im(d_{n+1})$ Homologieobjekt \\
- Kettenkomplex $(C_*, d_*)$ exakt $\rightarrow$ alle $H_n(C)$ verschwinden \\
- $Kom(A)$ Kategorie der Kettenkomplexe abelscher Kategorien \\
- Morphismus $u : C \mapsto D$ von Kettenkomplexen $\rightarrow$ Familie von Morphsimen $u_n : C_n \mapsto D_n$, sodass \\
- u bildet $H_n(C) \mapsto H_n(D)$ ab $\rightarrow$ quasi-Isomorphismus, falls alle $H_n(C) \mapsto H_n(D)$ Isomorphismen sind \\
\newpage

 WAS IST EINE ABGELEITETE KATEGORIE\\

{\large Sei A abelsche Kategorie und Kom(A) die Kategorie der Komplexe über A.
Dann $\exists$ Kategorie D(A) und ein Funktor $Q: Kom(A) \mapsto D(A):$}
\begin{description}
    \item[i) $Q(f) \: ist \: Isomorphismus \: fuer \: jeden \: quasi-Isomorphismus \: f: A \mapsto B$:]
        \[H_n(A_0) \mapsto H_n(B_0) \: ist \: Isomorphismus\]
    \item[ii) $Fuer \: jeden \: Funktor \: F : Kom(A) \mapsto D \: gilt$:]
        \[\exists! Funktor \: G : D(A) \mapsto D: F = G \circ D\]
\end{description}
D(A) heisst abgeleitete Kategorie von A. A heisst halbeinfach, falls jedes exakte Triple
T in A isomorph zu $0 \mapsto X \mapsto X \oplus Y \mapsto Y \mapsto 0$ ist. 

AUSSAGEN UEBER EXISTENZ \\
\newline
- test
- B Kategorie, S Klasse von Morphsimen in B \\
- Konstruiere Kategorie $B[S^{-1}] =: B_S$ und Funktor $Q : B \mapsto B_S$ \\
\newline
- $Obj(B_S) = Obj(B)$\\
- Konstruiere Morphismen in $B_S$
\begin{description}
    \item[i) $Variable \: x_s \: fuer \: jedes \: s \in S$]
    \item[ii) $gerichteten \: Graph \: \Gamma$:]
    \item[iii) $Pfad \: in \: \Gamma \: endliche \: Folge \: von \: Kanten$]
    \item[iv) $Morphismus \: in \: B_S \: ist \: Aequivalenzklasse \: von \: Pfaden \: in \: \Gamma$]
    \item[v) $Komposition \: von \: Morphismen \: \rightarrow \: entsprechende \: Pfade \: verbinden $]
\end{description}
- Zum Funktor $Q : B \mapsto B_S$
\begin{description}
    \item[i) $Q \: bildet \: Morphismus \: X \mapsto Y \: in \: Klasse \: der \: Pfade \: X \mapsto Y \: ab$]
    \item[ii) $\forall s \in S : Q(s) \: ist \: Isomorphismus \: in \: B_S$]
    \item[iii) $Funktor \: F : B \mapsto B'. Definiere \: G : B_S \mapsto B' \: mit \: F = G \circ Q \: durch$]
        \[G(X) = F(X), \: X \in Obj(B_S) = Obj(B)\]
        \[G(f) = F(f), \: f \in Mor(B)\]
        \[G(x_s) = F(s^{-1}), \: s \in S\]
\end{description}
WIE KANN DEREN STRUKTUR STUDIERT WERDEN\\

{\large Idee: Studiere zyklische Komplexe}
\begin{description}
    \item[i) $Komplex \: ZK \: heisst \: zyklisch, \: falls \: alle \: Differentiale \: null \: sind$]
    \item[ii) $Es \: gilt: Kom_0(A) \subset Kom(A) \: ist \: vollstaendige \: Unterkategorie$]  
    \item[iii) $Kom_o(A) \cong \prod_{n=-\infty}^\infty A_n$]
    \item[iv) $i : Kom(A) \mapsto Kom_0(A), \: h : Kom(A) \mapsto Kom_0(A)$:]
        \[h((K^n,d^n) = (H^n(K',0)))\]
        \[h(f:K' \mapsto L') = H^n(f)\]
    \item[v) $h \: kann  \: ueber \: D(A) \: faktorisiert \: werden$:]
    \item[vi) $Abelsche \: Kategorie \: A \: heisst \: halbeinfach, \: falls \: fuer \: jedes \: exakte \: Triple \: T$:]
        \[T \cong 0 \mapsto X \mapsto X \oplus Y \mapsto Y \mapsto 0\]
%   \item[vii) $Ist \: abelsche \: Kategorie \: A \: halbeinfach$:]
        \[Funktor \: D(A) \mapsto Kom_0(A) \: ist \: Aequivalenz \: von \: Kategorien\]
\end{description}
$\rightarrow$ Studiere Struktur ueber Isomorphismen.

\newpage

 WAS IST EINE LOKALISIERUNG\\

{\large Sei B beliebige Kategorie. $S \in Mor(B)$ Klasse von Morphismen heisst Lokalisierung, falls gilt}
\begin{description}
    \item[i) $S \: ist \: abgeschlossen$:]
        \[X \in Obj(B) \Rightarrow id_X \in S\]
        \[s,t \in S \Rightarrow s \circ t \in S\]
    \item[ii) $\forall f \in Mor(B), s \in S \: \exists g \in Mor(B), t \in S$:]
    \item[iii) $Seien \: f,g : X \mapsto Y$:]
        \[\exists s \in S: \: sf = sg \equiv \exists t \in S: \: ft = gt\]
\end{description}

WIE SIEHT $B[S^-1]$ AUS\\

{\large Sei S Lokalisierungsklasse von Kategorie B. Dann kann $B[S^{-1}] =: B_S$ beschrieben werden durch}
\begin{description}
    \item[i) $Obj(B_S) = Obj(B)$]
    \item[ii) $X \mapsto Y \in B_S \: ist \: z.B. \: Diagramm \: (s,f) \: in \: B$:]
        \[(s,f) \sim (t, g) \: falls\]
    \item[iii) $Komposition \: von \: Morphismen \: durch \: (s,f), \: (t,g) \: ist \: Klasse \: von \: (st',gf')$:]
\end{description}

WANN IST $B[S^{-1}] < C[S^{-1}]$

 {\large Sei $B \subset C$ vollstaendige Unterkategorie, S Lokalisierungsklasse von C}
\begin{description}
    \item[i) $S_B = S \cap Mor(B) \: Lokalisierungssystem \: in \: B$]
    \item[ii) $\forall s \in S : X' \mapsto X \in Obj(B) \exists f : X'' \mapsto X'$:]
        \[sf \in S, \: X'' \in Ob(B)\]
    \item[iii) $wie \: ii) \: Pfeile \: umgekehrt$]   
\end{description}
Gilt i) und ii) oder iii) $\rightarrow B[S_B^{-1}] < C[S^{-1}]$ vollstaendig, $k : B[S_B^{-1}] \mapsto C[S^{-1}]$ volltreu. 

\end{document}
